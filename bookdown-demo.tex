% Options for packages loaded elsewhere
\PassOptionsToPackage{unicode}{hyperref}
\PassOptionsToPackage{hyphens}{url}
%
\documentclass[
]{book}
\usepackage{amsmath,amssymb}
\usepackage{iftex}
\ifPDFTeX
  \usepackage[T1]{fontenc}
  \usepackage[utf8]{inputenc}
  \usepackage{textcomp} % provide euro and other symbols
\else % if luatex or xetex
  \usepackage{unicode-math} % this also loads fontspec
  \defaultfontfeatures{Scale=MatchLowercase}
  \defaultfontfeatures[\rmfamily]{Ligatures=TeX,Scale=1}
\fi
\usepackage{lmodern}
\ifPDFTeX\else
  % xetex/luatex font selection
\fi
% Use upquote if available, for straight quotes in verbatim environments
\IfFileExists{upquote.sty}{\usepackage{upquote}}{}
\IfFileExists{microtype.sty}{% use microtype if available
  \usepackage[]{microtype}
  \UseMicrotypeSet[protrusion]{basicmath} % disable protrusion for tt fonts
}{}
\makeatletter
\@ifundefined{KOMAClassName}{% if non-KOMA class
  \IfFileExists{parskip.sty}{%
    \usepackage{parskip}
  }{% else
    \setlength{\parindent}{0pt}
    \setlength{\parskip}{6pt plus 2pt minus 1pt}}
}{% if KOMA class
  \KOMAoptions{parskip=half}}
\makeatother
\usepackage{xcolor}
\usepackage{longtable,booktabs,array}
\usepackage{calc} % for calculating minipage widths
% Correct order of tables after \paragraph or \subparagraph
\usepackage{etoolbox}
\makeatletter
\patchcmd\longtable{\par}{\if@noskipsec\mbox{}\fi\par}{}{}
\makeatother
% Allow footnotes in longtable head/foot
\IfFileExists{footnotehyper.sty}{\usepackage{footnotehyper}}{\usepackage{footnote}}
\makesavenoteenv{longtable}
\usepackage{graphicx}
\makeatletter
\def\maxwidth{\ifdim\Gin@nat@width>\linewidth\linewidth\else\Gin@nat@width\fi}
\def\maxheight{\ifdim\Gin@nat@height>\textheight\textheight\else\Gin@nat@height\fi}
\makeatother
% Scale images if necessary, so that they will not overflow the page
% margins by default, and it is still possible to overwrite the defaults
% using explicit options in \includegraphics[width, height, ...]{}
\setkeys{Gin}{width=\maxwidth,height=\maxheight,keepaspectratio}
% Set default figure placement to htbp
\makeatletter
\def\fps@figure{htbp}
\makeatother
\setlength{\emergencystretch}{3em} % prevent overfull lines
\providecommand{\tightlist}{%
  \setlength{\itemsep}{0pt}\setlength{\parskip}{0pt}}
\setcounter{secnumdepth}{5}
\usepackage{booktabs}
\usepackage{amsthm}
\makeatletter
\def\thm@space@setup{%
  \thm@preskip=8pt plus 2pt minus 4pt
  \thm@postskip=\thm@preskip
}
\makeatother
\ifLuaTeX
  \usepackage{selnolig}  % disable illegal ligatures
\fi
\usepackage[]{natbib}
\bibliographystyle{apalike}
\IfFileExists{bookmark.sty}{\usepackage{bookmark}}{\usepackage{hyperref}}
\IfFileExists{xurl.sty}{\usepackage{xurl}}{} % add URL line breaks if available
\urlstyle{same}
\hypersetup{
  pdftitle={MA-500: Introduction to R},
  pdfauthor={Regina-Mae Dominguez - Fall 2024},
  hidelinks,
  pdfcreator={LaTeX via pandoc}}

\title{MA-500: Introduction to R}
\author{Regina-Mae Dominguez - Fall 2024}
\date{}

\begin{document}
\maketitle

{
\setcounter{tocdepth}{1}
\tableofcontents
}
\hypertarget{course-introduction}{%
\chapter*{Course Introduction}\label{course-introduction}}
\addcontentsline{toc}{chapter}{Course Introduction}

Moodle is the platform where you'll upload and submit homework assignments, as well as access other course-related materials. This \texttt{markdown} site serves as a central hub for all course information, including notes, guides, and resources.

\hypertarget{installing-r}{%
\section*{Installing R}\label{installing-r}}
\addcontentsline{toc}{section}{Installing R}

To install R, begin by visiting the Comprehensive R Archive Network (CRAN) here: \url{https://cran.r-project.org/}. Select and download the appropriate R binary package for your operating system--- whether Windows, macOS, or Linux. For Mac users, ensure you install the correct package binary associated with your processor (e.g., Intel or Apple Silicon).

\hypertarget{installing-rstudio}{%
\section*{Installing RStudio}\label{installing-rstudio}}
\addcontentsline{toc}{section}{Installing RStudio}

RStudio is the most widely used integrated development environment (IDE) for R programming. You can download the free version here: \url{https://posit.co/downloads/}. Whie you have the option to use alternative IDEs, such as VS Code with the Rtools extention or the base R GUI, it is recommended to use RStudio as the course material will primarily be demonstrated using this IDE. This will ensure you can easily follow along with the course content!

\hypertarget{installation}{%
\chapter*{Installation}\label{installation}}
\addcontentsline{toc}{chapter}{Installation}

\hypertarget{installing-r-1}{%
\section*{Installing R}\label{installing-r-1}}
\addcontentsline{toc}{section}{Installing R}

To install R, begin by visiting the Comprehensive R Archive Network (CRAN) here: \url{https://cran.r-project.org/}. Select and download the appropriate R binary package for your operating system--- whether Windows, macOS, or Linux. For Mac users, ensure you install the correct package binary associated with your processor (e.g., Intel or Apple Silicon).

\hypertarget{installing-rstudio-1}{%
\section*{Installing RStudio}\label{installing-rstudio-1}}
\addcontentsline{toc}{section}{Installing RStudio}

RStudio is the most widely used integrated development environment (IDE) for R programming. You can download the free version here: \url{https://posit.co/downloads/}. Whie you have the option to use alternative IDEs, such as VS Code with the Rtools extention or the base R GUI, it is recommended to use RStudio as the course material will primarily be demonstrated using this IDE. This will ensure you can easily follow along with the course content!

  \bibliography{book.bib,packages.bib}

\end{document}
